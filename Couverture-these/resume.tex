\ifodd\value{page}\hbox{}\newpage\fi

\markboth{}{}
% Plus petite marge du bas pour la quatrième de couverture
% Shorter bottom margin for the back cover
\newgeometry{inner=30mm,outer=20mm,top=40mm,bottom=20mm}

%insertion de l'image de fond du dos (resume)
%background image for resume (back)
\backcoverheader

% Switch font style to back cover style
\selectfontbackcover{ % Font style change is limited to this page using braces, just in case

\titleFR{Méthode Deep Learning and analyse Double Calorimétrique pour la mesure de haute précision des paramètres d'oscillation des neutrinos dans JUNO}

\keywordsFR{Neutrinos; expérience JUNO; Deep Learning; reconstruction d'IBD; oscillations des neutrinos; double calorimetrie}

%\abstractFR{La Chromodynamique Quantique (QCD) prédit l'existence d'un état de la matière appelé Plasma de Quarks et de Gluons (PQG) dans des conditions extrêmes de température et de pression, qui peut être produit lors de collisions d'ions lourds. Une observable du PQG est la suppression des quarkonia (états liés de quark-antiquark lourds), qui est définie par une plus faible production de ces états en présence de PQG par rapport à la production en l'absence de plasma. Ces dernières années, un effort signicatif a été réalisé d'un point de vue théorique vers une description dynamique des quarkonia au sein du Plasma de Quarks et de Gluons, à l'aide du formalisme des systèmes quantiques ouverts. Dans ce cadre, il est possible d'obtenir une description en temps réel d'un système quantique (ici un quarkonium) en interaction avec un bain thermique (le PQG) en étudiant la matrice densité réduite du système. Cette thèse étudie la dynamique d'états quarkonium en résolvant une équation maîtresse quantique basée sur l'approche de Blaizot \& Escobedo. Plus précisement, cette équation est résolue numériquement directement pour la première fois dans le cas d'un PQG statique et dans le cas d'un PQG se refroidissant. Les populations d'états quarkonium sont étudiées et la validité d'approximations semi-classiques amenant à des équations de Langevin est examinée.}
\abstractFR{JUNO est un observatoire de neutrinos à scintillateur liquide, polyvalent et medium baseline (environ 52 km), situé en Chine. Ses principaux objectifs sont de mesurer les paramètres d'oscillation $\theta_{12}$, $\Delta m^2_{21}$ et $\Delta m^2_{31}$ avec une précision au pour-mille et de déterminer l'ordre des masses des neutrinos avec un niveau de confiance de 3$\sigma$. Atteindre ces objectifs nécessite une résolution énergétique sans précédent de $3\% / \sqrt{\mathrm{E(MeV)}}$ avec cette technologie. Cela demande une compréhension approfondie des divers effets au sein du détecteur. Le système de double calorimetrie, composé de deux systèmes de mesure distincts observant le même événement, permet non seulement une calibration mais aussi une détection des effets du détecteur avec une grande précision, comme démontré dans cette thèse. Le Deep Learning, un outil de plus en plus utilisé en physique expérimentale, joue un rôle crucial dans cet effort. Dans cette thèse, je présente le développement, l'application et l'analyse des techniques de Deep Learning pour la reconstruction d'évèvements dans l'expérience JUNO.}


\titleEN{Deep learning methods and Dual Calorimetric analysis for high precision neutrino oscillation measurements at JUNO}

\keywordsEN{Neutrinos; JUNO experiment; Deep learning; IBD reconstruction; neutrinos Oscillation; dual Calorimetry}

%\abstractEN{Quantum chromodynamics (QCD) predicts the existance of a state of matter called the Quark-Gluon Plasma (QGP) at extreme temperature and density, which can be produced in heavy ion collisions. One of the QGP observables is the so-called quarkonia (heavy quark-antiquark bound states) suppression which is defined by a smaller production of quarkonia states in presence of QGP compared to the production in absence of plasma. In recent years, a significant theoretical effort has been made towards a dynamical description of quarkonia inside the Quark-Gluon Plasma , using the open quantum systems formalism. In this framework, one can get a real-time description of a quantum system (here a quarkonium) in interaction with a thermal bath (the QGP) by integrating out the bath degrees of freedom and studying the system reduced density matrix. This thesis investigates the dynamics of quarkonium states by resolving a quantum master equation based on the approach of Blaizot \& Escobedo. More precisely, this equation is resolved numerically directly for the first-time in both a static and a cooling QGP. The populations of quarkonium states over time are studied and the validity of semi-classical approximations leading to Langevin equations is investigated.}
\abstractEN{JUNO is a multipurpose, medium-baseline ($\sim$52 km) liquid scintillator neutrino observatory located in China. Its primary objectives are to measure the oscillation parameters $\theta_{12}$, $\Delta m^2_{21}$, and $\Delta m^2_{31}$ with per mil precision and to determine the neutrino mass ordering at a 3$\sigma$ confidence level. Achieving these goals requires an unprecedented energy resolution of $3\% / \sqrt{\mathrm{E(MeV)}}$ with this technology. This demands a comprehensive understanding of the various effects within the detector. The Dual Calorimetry system-two distinct measurement systems observing the same event-enables not only high-precision calibration but also detection of detector effects, as demonstrated in this thesis. Deep learning, an increasingly powerful tool in physics, plays a critical role in this effort. In this thesis, I present the development, application, and analysis of Deep Learning techniques for reconstruction in the JUNO experiment.}
}

% Rétablit les marges d'origines
% Restore original margin settings
\restoregeometry
