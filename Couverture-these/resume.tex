\markboth{}{}
% Plus petite marge du bas pour la quatrième de couverture
% Shorter bottom margin for the back cover
\newgeometry{inner=30mm,outer=20mm,top=40mm,bottom=20mm}

%insertion de l'image de fond du dos (resume)
%background image for resume (back)
\backcoverheader

% Switch font style to back cover style
\selectfontbackcover{ % Font style change is limited to this page using braces, just in case

\titleFR{Description théorique de la dynamique des quarkonia dans le plasma de quarks et de gluons au moyen d'une approche de type équation maîtresse quantique}

\keywordsFR{Dynamique des quarkonia; systèmes quantiques ouverts; plasma de quarks et de gluons; équation maîtresse quantique}

%\abstractFR{La Chromodynamique Quantique (QCD) prédit l'existence d'un état de la matière appelé Plasma de Quarks et de Gluons (PQG) dans des conditions extrêmes de température et de pression, qui peut être produit lors de collisions d'ions lourds. Une observable du PQG est la suppression des quarkonia (états liés de quark-antiquark lourds), qui est définie par une plus faible production de ces états en présence de PQG par rapport à la production en l'absence de plasma. Ces dernières années, un effort signicatif a été réalisé d'un point de vue théorique vers une description dynamique des quarkonia au sein du Plasma de Quarks et de Gluons, à l'aide du formalisme des systèmes quantiques ouverts. Dans ce cadre, il est possible d'obtenir une description en temps réel d'un système quantique (ici un quarkonium) en interaction avec un bain thermique (le PQG) en étudiant la matrice densité réduite du système. Cette thèse étudie la dynamique d'états quarkonium en résolvant une équation maîtresse quantique basée sur l'approche de Blaizot \& Escobedo. Plus précisement, cette équation est résolue numériquement directement pour la première fois dans le cas d'un PQG statique et dans le cas d'un PQG se refroidissant. Les populations d'états quarkonium sont étudiées et la validité d'approximations semi-classiques amenant à des équations de Langevin est examinée.}
\abstractFR{La Chromodynamique Quantique (QCD) prédit l'existence d'un état de la matière appelé Plasma de Quarks et de Gluons (PQG) dans des conditions extrêmes de température et de pression, qui peut être produit lors de collisions d'ions lourds. Une observable du PQG est la suppression des quarkonia (états liés de quark-antiquark lourds), qui est définie par une plus faible production de ces états en présence de PQG par rapport à la production en l'absence de plasma. Ces dernières années, un effort signicatif a été réalisé d'un point de vue théorique vers une description dynamique des quarkonia au sein du Plasma de Quarks et de Gluons, à l'aide du formalisme des systèmes quantiques ouverts. Cette thèse étudie la dynamique d'états quarkonium en résolvant une équation maîtresse quantique basée sur l'approche de Blaizot \& Escobedo. Plus précisement, cette équation est résolue numériquement directement pour la première fois dans le cas d'un PQG statique et dans le cas d'un PQG se refroidissant. Les populations d'états quarkonium sont étudiées et la validité d'approximations semi-classiques amenant à des équations de Langevin est examinée.}


\titleEN{Theoretical description of quarkonium dynamics in the quark gluon plasma with a quantum master equation approach}

\keywordsEN{Quarkonia dynamics; open quantum systems; quark-gluon plasma; quantum master equation}

%\abstractEN{Quantum chromodynamics (QCD) predicts the existance of a state of matter called the Quark-Gluon Plasma (QGP) at extreme temperature and density, which can be produced in heavy ion collisions. One of the QGP observables is the so-called quarkonia (heavy quark-antiquark bound states) suppression which is defined by a smaller production of quarkonia states in presence of QGP compared to the production in absence of plasma. In recent years, a significant theoretical effort has been made towards a dynamical description of quarkonia inside the Quark-Gluon Plasma , using the open quantum systems formalism. In this framework, one can get a real-time description of a quantum system (here a quarkonium) in interaction with a thermal bath (the QGP) by integrating out the bath degrees of freedom and studying the system reduced density matrix. This thesis investigates the dynamics of quarkonium states by resolving a quantum master equation based on the approach of Blaizot \& Escobedo. More precisely, this equation is resolved numerically directly for the first-time in both a static and a cooling QGP. The populations of quarkonium states over time are studied and the validity of semi-classical approximations leading to Langevin equations is investigated.}
\abstractEN{Quantum chromodynamics (QCD) predicts the existance of a state of matter called the Quark-Gluon Plasma (QGP) at extreme temperature and density, which can be produced in heavy ion collisions. One of the QGP observables is the so-called quarkonia (heavy quark-antiquark bound states) suppression which is defined by a smaller production of quarkonia states in presence of QGP compared to the production in absence of plasma. In recent years, a significant theoretical effort has been made towards a dynamical description of quarkonia inside the Quark-Gluon Plasma , using the open quantum systems formalism. This thesis investigates the dynamics of quarkonium states by resolving a quantum master equation based on the approach of Blaizot \& Escobedo. More precisely, this equation is resolved numerically directly for the first-time in both a static and a cooling QGP. The populations of quarkonium states over time are studied and the validity of semi-classical approximations leading to Langevin equations is investigated.}
}

% Rétablit les marges d'origines
% Restore original margin settings
\restoregeometry
