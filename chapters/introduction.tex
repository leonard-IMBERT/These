\chapter*{Introduction}
\addcontentsline{toc}{chapter}{Introduction}
\markboth{Introduction}{Introduction}

The Standard Model of particle physics (SM) has been remarkably successful at accounting for, or predicting experimental observations in the laboratory.
However, it is the subject of several limitations. For instance, it provides a mechanism to explain the existence of mass but can't predict the peculiar pattern followed by fermion masses.
The same applies to CP violation. The SM predicts its existence but not the amplitude necessary to explain the baryonic asymmetry of the Universe. For such reasons, one can assume the SM is the manifestation of a more fundamental physics, Beyond the Standard Model (BSM).

Neutrino physics is a window on BSM. Indeed, the mass of known neutrinos is at least 5 order of magnitudes below that of the lightest fermion, which further deepens
the issue of fermion mass generation.
Some solutions have implication on the nature of neutrinos -- dirac or majorana fermions ?  --  which one of the big unknowns in this domain. Additional neutrinos beyond the three presently known shall also be considered. The way neutrinos mix flavor to make neutrino oscillation possible is also unexplained.
This is one of the tasks of BSM models to answer such questions. Before that, a good part of the World experimental program in the 10 coming years is to complete the exploration of 3-neutrino physics by answering mainly two questions :
does CP violation exist the lepton system ? What is the Neutrino Mass ordering (NMO) ?
An introduction to neutrino physics will be given in Chapter \ref{sec:neutrino}.

\hfill

The Jiangmen Underground Neutrino Observatory (JUNO), currently under construction in China, aims to address these questions, particularly the determination of the NMO. JUNO's approach is to study reactor antineutrinos emitted from nearby nuclear power plants. By precisely measuring the energy spectrum of these antineutrinos after oscillation, JUNO seeks to detect the subtle interference patterns in the spectrum that are sensitive to the NMO. The ability to achieve this requires unprecedented precision in both the energy resolution and the calibration of the detector’s response to neutrino events. JUNO is expected to start data collection in 2025, with the goal of determining the NMO at a significance level of 3-4$\sigma$ after six years of data taking.
At the heart of JUNO’s experimental design is its dual calorimetry system, comprising two separate sets of photomultipliers—large (LPMT) and small (SPMT) PMTs—that allow for independent energy measurements of the same events. This dual system is not only essential for improving energy resolution but also for providing cross-checks that ensure systematic uncertainties are well-understood and minimized. Achieving JUNO’s goals depends on this dual calorimetry system, as it will enable precise reconstruction of the energy spectrum and the identification of potential discrepancies between the two systems.


\hfill

Another emerging area of importance in particle physics experiments is the application of machine learning (ML) techniques. Over the past decade, ML methods, particularly deep learning, have been increasingly used to tackle complex problems in event classification, reconstruction, and even data generation like the High luminosity LHC Upgraded experiments. Performant online reconstruction, critical for the trigger systems of such
 experiments, is another example. The complexity of the data and the required precision in experiments such as JUNO make ML an attractive tool. In particular, Neural Networks (NNs) and other advanced ML models have shown potential for improving the accuracy of energy reconstruction and other key analysis tasks. However, for the results obtained using ML methods to be trusted by the scientific community, the reliability of these methods must be rigorously demonstrated.
An introduction to ML, and in particular Neural Network (NN) is given in Chapter \ref{sec:ml}.

\hfill

This thesis was performed in the framework of the Neutrino group at Subatech, since October 2021. The exploratory works reported in this manuscript addresses the subjects mentioned above, in the particular context of the measurement by JUNO of the reactor antineutrino
oscillation to determine the NMO.
Before the start of this thesis, several ML energy reconstruction algorithms -- Boosted Decision Trees (BDT), Fully Connected Neural Networks (FCNN), Convolutional Neural Networks (CNNs) and Graph Neural Networks (GNNs) --  had already been developed within
the collaboration. Their performance seems to match that of the classical algorithm but not to do convincingly better. We have explored a possibility
to do better by developing a GNN with an innovative architecture tailored to the JUNO experiment. Before that, we developed a CNN for the reconstruction of the anti-neutrino
energy using only JUNO's small PMTs system.
This CNN is useful in particular in Chapter \ref{sec:joint_fit} as there is official SPMT only reconstruction in the collaboration yet. These algorithms are described in Chapters \ref{sec:jcnn} and \ref{sec:jgnn}.

We have been the first in JUNO to address the issue of ML reliability.
% We have followed two paths for that.
% First, a simple approach is to compare event per event the results obtained by various algorithms, to find discrepancies, and more generally differences or common points in the way detector's information is used.
% This requires to implement in JUNO's official software algorithms traditionally developed standalone, as well as the necessary software tools. This was our contribution there.
We explore in this thesis the feasibility of an Adversarial Neural Network (ANN) to generate (and therefore identify) scenarios of discrepancies
between raw data in the real detector and in the detector's simulation.
The focus here is on discrepancies that could alter JUNO's results on NMO, but are too subtle to be detected via usual data/MC comparisons in control samples.
This is presented in Chapter \ref{sec:janne}.


\hfill

We have already mentioned earlier it is crucial for JUNO to understand its energy scale with a good precision. This is the raison d'être of the existence of two calorimetric readout systems : the large (LPMT) and small (SPMT) photomultipliers systems.
It allows Dual Calorimetry techniques to constrain our understanding of the reconstruction. The last subject of this thesis explores for the first time one of them : the Dual Calorimetry with neutrino oscillation,
which leverages potential discrepancies between the oscillation analyses performed with each system.
Our work on this is described in Chapter \ref{sec:joint_fit}.
It was also the occasion of technical developments on the analysis framework used at Subatech. These improvements will be very useful for future analyses of the group, beyond Dual calorimetry.
