\documentclass[../main.tex]{subfiles}
\graphicspath{{\subfix{..}}}

\begin{document}
\chapter*{Résumé}
\addcontentsline{toc}{chapter}{Résumé}
\markboth{Résumé}{Résumé}
\begin{otherlanguage}{french}

Le Modèle Standard (SM) de la physique des particules constitue une réalisation majeure de la science moderne, capable de décrire et de prédire de nombreux phénomènes expérimentaux. Cependant, des limitations subsistent. Le SM propose un mécanisme de génération des masses, mais ne peut expliquer la différence entre les masses des fermions. De même pour la violation Charge-Parité (CP), qui est prédite, mais le SM ne prédit pas une amplitude suffisante  pour justifier l'asymétrie matière-antimatière de l'Univers. Ces limites suggèrent l'existence d'une physique fondamentale au-delà du SM, connue sous le nom de ``physique au-delà du Modèle Standard'' (BSM).

La physique des neutrinos offre une perspective unique sur cette physique BSM. En effet, la masse des neutrinos est à minima cinq ordres de grandeur inférieure à celle du fermion le plus léger, soulevant des questions sur le mécanisme de génération des masses. La réponse auras un impact important sur une autre grande question sur la nature des neutrinos: sont-ils des fermions de Dirac ou de Majorana? L'amplitude des mélanges de saveurs est aussi un des paramètres libres du SM, une théorie BSM pourrais être amené à expliquer la valeur de ces paramètres de mélange. Avant de pouvoir produire un tel modèle, le neutrino doit être caractérisé, et les grandes expériences de physiques des 10 prochaines années auront pour objectif de répondre à deux grandes questions: Existe-t-il une violation CP dans le secteur leptonique? Et quel est l'ordre des masses des neutrinos?

\hfill

C'est pour répondre à ces questions qu'a été développée l'expérience Jiangmen Underground Neutrino Observatory (JUNO) -- Observatoire de Neutrino Souterrain de Jiangmen. Ce détecteur au liquide scintillant de 20 kilotonnes en cours de construction en Chine a pour objectif principal la détermination de l'Ordre des Masses des Neutrinos (NMO) avec une signifiance de 3$\sigma$ après 6,5 ans de prise de donnée et la mesure des paramètres d'oscillation $\theta_{12}$, $\Delta m^2_{21}$ et $\Delta m^2_{31}$ au millième près.

Au c\oe{}ur du design de l'expérience JUNO se trouve le système de double calorimétrie, composé de deux systèmes de tubes photo-multiplicateurs: les grands (LPMT) et petits (SPMT) photomultiplicateurs. Avoir deux systèmes permet deux mesures indépendantes des mêmes évènements. La dualité de ces systèmes ne permet pas seulement d'améliorer la résolution en énergie, mais est crucial pour la calibration afin de valider que les incertitudes systématiques sont correctement évaluées et minimisées. JUNO auras besoin de ce système de double calorimétrie pour une reconstruction précise du spectre en énergie et l'identification de potentielles différences entre les deux systèmes

Un autre domaine émergeant d'importance dans les expériences de physique des particules est l'utilisation de l'intelligence artificielle (IA). Les méthodes d'IA, en particulier l'apprentissage profond, sont de plus en plus utilisées pour résoudre des problèmes complexes dans la classification d'événements, la reconstruction, et même la génération de données. La reconstruction en temps réelle, essentielle pour les systèmes de déclenchement de ces expériences, en est un autre exemple, nécessitant performance et haute vitesse de reconstruction. La complexité des données et la précision requise dans des expériences comme JUNO font de l'IA un outil attrayant. En particulier, les réseaux neuronaux (NN) et d'autres modèles avancés d'IA ont montré leur potentiel pour améliorer la précision de la reconstruction énergétique et d'autres tâches clés d'analyse. Cependant, pour que les résultats obtenus à l'aide des méthodes d'IA soient acceptés par la communauté scientifique, la fiabilité de ces méthodes doit être rigoureusement démontrée.

Cette thèse a été réalisée au sein du groupe Neutrino de Subatech et a débuté en octobre 2021. Les travaux exploratoires rapportés dans ce manuscrit abordent les sujets mentionnés ci-dessus, dans le contexte particulier de la mesure, par JUNO, de l'oscillation des antineutrinos de réacteur afin de déterminer l'ordre de masse des neutrinos (NMO).
Avant le début de cette thèse, plusieurs algorithmes de reconstruction énergétique basés sur l'apprentissage automatique -- arbres de décision boostés (BDT), réseaux neuronaux entièrement connectés (FCNN), réseaux neuronaux convolutionnels (CNN) et réseaux neuronaux graphiques (GNN) -- avaient déjà été développés au sein de la collaboration. Leurs performances semblent correspondre à celles de l'algorithme classique, mais sans apporter une amélioration significative ou convaincante.

\hfill

Dans le Chapitre \ref{sec:neutrino}, j'effectue un bref rappel du modèle standard de la physique des particules et discute notamment ses limites. La suite de ce chapitre se concentre sur les neutrinos, leurs interactions avec la matière, leur oscillation et leur phénoménologie. Sont rappelé au sein de chapitre, l'état de l'art des mesures des paramètres d'oscillation. La dernière partie de ce chapitre présente les questions ouvertes autour de ces particules. Parmi ces questions, une relève particulièrement d'intérêt pour cette thèse: L'ordre des masses des neutrinos.


Le deuxième chapitre de cette thèse se concentre sur la description de l'expérience JUNO, ses objectifs physique et les moyens mis en \oe{}vre pour y parvenir. Un soin particulier est porté à la description du système de détection, au c\oe{}r du sujet de la reconstruction, sujet des Chapitres \ref{sec:jcnn}, \ref{sec:jgnn} et \ref{sec:janne}. Finalement est introduit la méthode d'analyse du spectre neutrino, sujet du Chapitre \ref{sec:joint_fit}.


Le troisième chapitre est une introduction aux méthodes d'intelligence artificielle utilisées au cours de cette thèse. Est notamment introduit les principes fondamentaux de modèles en couche et de descente de gradient. Les technologies de réseaux de neurones convolutifs, de réseaux de neurones en graphe et de réseaux de neurones adversoriels sont présenté et imagé par des cas d'exemples physiques. La dernière partie de ce chapitre présente l'état de l'art de la reconstruction d'évènements IBD dans l'expérience JUNO, tant par des méthodes statistique classique tel que la minimisation de la vraisemblance que par des techniques d'intelligence artificielle.


\hfill

Les chapitres suivants présentent mes travaux, en commençant, dans le Chapitre \ref{sec:jcnn}, par le développement d'un réseau de neurone convolutif pour la reconstruction d'évènements IBD en utilisant seulement le système des petits PMTs. Un algorithme classique basé sur une méthode de minimisation de la vraisemblance avais déjà été développé au sein de l'équipe à Subatech et a servis d'algorithme de référence afin de comparer les résultats du réseau de neurones. Les performances de notre réseau sont comparables à celles obtenues par la méthode classique, mais nous observons une légère décorrélation dans l'erreur des deux méthodes. Cette décorrélation nous permets, via une méthode de combinaison d'estimateur développé durant cette thèse, d'obtenir la performance atteignable par une méthode qui utiliserait les points fort des deux méthodes. La conclusion de ce chapitre discute des performances et des limites de l'utilisation d'un réseau convolutif planaire dans une expérience de topologie sphérique et sert d'introduction au chapitre suivant dédié au développement de réseaux de neurones en graphe.


Le Chapitre \ref{sec:jgnn} présente mes travaux dans le développement d'architecture neuronale en graphe pour la reconstruction d'évènement IBD. Le début de ce chapitre présente la problématique qui a motivé l'utilisation de réseau de neurones en graphe: la topologie sphérique de JUNO, la volonté de pouvoir mixer au sein de la même représentation des données de bas niveau tel que la charge intégrée et le temps de collecte de chaque PMT et des informations de plus haut niveau tel que des analyses en harmoniques sphériques par exemple. Le détail de ces analyses en harmoniques sphérique sont présentes dans l'annexe \ref{sec:annex:jgnn:harms}. Toutes ces problématique nous mènes à l'utilisation d'une représentation en graphe de l'expérience. J'introduis dans ce chapitre le principe de graphe hétérogène -- tous les n\oe{}uds du graphe ne possède pas les mêmes caractéristiques topologiques -- et le fait que le traitement de graphe hétérogène n'est pas encore développé au sein de la communauté de l'intelligence artificiel. Je propose dans ce chapitre un algorithme de ``Message Passing'' adapté à un graphe hétérogène et présente un modèle utilisant cet algorithme. Les performances de reconstruction de ce modèle est comparé à l'état de l'art de la reconstruction dans JUNO présenté dans le Chapitre \ref{sec:ml}. Les performances de notre modèle ne sont pas au niveau des algorithmes existant et la conclusion passe en revue les potentiels causes de ce manque de performance et propose des solutions pour de futurs développements.

Le Chapitre \ref{sec:janne} introduit une problématique que nous pensons au c\oe{}ur de l'usage de l'intelligence artificielle au sein de la physique des particules: la fiabilité des résultats. Afin de pouvoir tirer des conclusions scientifiques, il est crucial de pouvoir démontrer la fiabilité de nos modèles. Nous explorons dans ce chapitre une méthode permettant de sonder les potentiels faiblesse et biais de nos algorithmes en utilisant des réseaux de neurones génératif adversoriels. Le premier prototype présenté dans cette thèse ne répond pas encore aux exigences, mais ces travaux pourront servir de base de réflexion pour l'exploration de cette problématique.

Le dernier chapitre de résultats de cette thèse, le Chapitre \ref{sec:joint_fit}, présente une méthode d'analyse du spectre d'oscillation IBD, l'analyse double calorimétrique. Cette méthode se base sur l'exploitation des corrélations entre le spectre reconstruit par les LPMT et celui reconstruit par les SPMT afin de détecter de potentielles incohérences entre eux. Le principe de base est que les deux systèmes observent les mêmes évènements, mais que leur reconstruction sont indépendantes. Ainsi les fluctuations statistiques doivent être corrélé et nous pouvons tester la cohérence de ces corrélations entre ces deux spectres. Je commence par présenter les résultats de cette analyse en l'absence de perturbation sur le spectre LPMT puis en introduisant une déformation probable, une non-linéarité dans la reconstruction LPMT. Plusieurs tests statistiques ont été développés et leurs sensibilités sont présentés. D'après nos tests, si une perturbation de type non-linéarité d'amplitude $\alpha_{qnl} = 0.3\%$ -- l'ordre de grandeur garanti par la calibration --  est invisible dans le spectre LPMT, nos outils permettront de la détecter à 6 ans de statistique avec une p-value de 0.5\%. À la fin de ce chapitre, nous rappelons les limitations et hypothèses de ces travaux et explorons les moyens d'y répondre.

\hfill

En conclusion, j'ai exploré l'usage de techniques d'intelligence artificielles pour la reconstruction d'évènement IBD dans l'expérience JUNO. Les modèles présentés dans cette thèse ne sont par compétitifs en termes de performances avec les algorithmes classiques, mais dans le cas de la reconstruction SPMT indique l'existence d'un algorithme plus performant que ceux existants. Ces travaux ont aussi été l'occasion de développer une architecture d'intelligence artificielle exotique pour le traitement de graphe hétérogène.

La problématique de fiabilité des algorithmes d'intelligence artificielle est aussi abordé et le développement d'une méthode prototype pour cette problématique est présenté.

Finalement, une méthode d'analyse double calorimétrique est présenté et les tests effectués indiquent que cette méthode nous permettras de détecter de potentielles déformations dans le spectre LPMT.

Tous ces travaux contribuent à la fiabilité et la robustesse de la mesure de l'ordre des masses et la mesure de précision des paramètres d'oscillation par l'expérience JUNO dans les années à venir.

\end{otherlanguage}
\end{document}
