\chapter{Image recognition for IBD reconstruction with the SPMT system}
\label{sec:jcnn}
\begin{table}[ht]
  \centering
  \begin{tabular}{ | c | c | }
    \hline $N_{blocks}$ & {2, 3, 4} \\
    \hline $N_{channels}$ & {32, 64, 128} \\
    \hline
    \multirow{4}{*}{FCNN configuration} & 2 * 1024 \\
                                        & 2 * 2048 + 2 * 1024 \\
                                        & 3 * 2048 + 3 * 512 \\
                                        & 2 * 4096 \\
    \hline
    Loss & $E+V$, $E_r + V_r$ \\
    \hline

  \end{tabular}
\end{table}

\section{Motivations}

\begin{itemize}
  \item Promise of machine learning -> Exploit raw data
  \item Victor already done reco for SPMT
  \item Can CNN give similar results ? Better results ?
  \item Multiple reco methods good for reconstruction
  \item Comparison, difference in behavior
\end{itemize}

\section{Method and model}

\begin{itemize}
  \item JUNO is an array of sensor following a quasi uniform and istropic geometric repartition -> Basically pixels -> Image
  \item CNN is gud for image processing (cite a lot of things)
  \item Details the architecture (Inspired from VGG 16)
    \begin{itemize}
      \item Convolutional layers
      \item Pooling layers -> Twice the channels when pooling by 2 -> Keep the same "amount" of information
      \item Dropout (introduce overtraining, maybe introduce overfitting in ML chapter ?)
      \item Vectorization fed to FCDNN
    \end{itemize}
  \item Data is 240x240 images
    \begin{itemize}
      \item Following $\theta$ and $\phi$ distribution, explain the coordinate system of JUNO
      \item Optimized for $\approx$ 1 SPMT/pixel
      \item 1 Charge channel
      \item 1 Time channel
    \end{itemize}
  \item Discuss data format
    \begin{itemize}
      \item Empty pixel ? -> $Q = 0$, $T = 0$, what does it means/says ? 0 = no signal in a way
      \item Image distortion
        \begin{itemize}
          \item \textbf{Maybe speak of this in the conclusion ?} Could be done in two step:
          \item 1. Reconstruct $\theta$ and $\phi$
          \item 2. "Rotate" the image so the event is at the center of the image -> Prevent distortion + reconstruction E and R become pseudo rotational invariant (as they should be)
        \end{itemize}
      \item 1 Millions MC e+ events for training (900k for train, 50k for validation and 50k for test)
        \begin{itemize}
          \item MC for the moment, will need to retrain with mix of calibration data (Good question, is the CNN PID agnostic ?)
          \item 47 IBD/day -> 1M event is 21k days of data (for reference, 6 years of data is 94k events)
          \item Events are "optimistic"
            \begin{itemize}
              \item No pile-up
              \item w/o neutrons
              \item time window is decided by electronics
              \item We want to reconstruct the E from $\bar{\nu_e}$
              \item Difference between multiple E -> $E_{vis}$, $E_{rec}$, $E_k$
            \end{itemize}
        \end{itemize}
    \end{itemize}
\end{itemize}

\section{Results}

\begin{itemize}
  \item Comparison with victor results
  \item \textbf{More details when I'll look into the retrained data}
  \item Discuss of the differences
  \item Discuss of the principle of error decorelation
    \begin{itemize}
      \item Possible improvements
      \item Combining algorithms
      \item Average sum
    \end{itemize}
\end{itemize}

\section{Conclusion}
Intoduction next chapter
