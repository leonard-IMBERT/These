\chapter{The JUNO experiment}

The first idea of a medium baseline (~60 km) experiment, was explored in 2008 where it was demonstrated that the Neutrino Mass Ordering (NMO) could be determined by a medium baseline experiment if $\sin^2(2\theta_{13}) > 0.005$ without requirements on accurate information of reactor antineutrino spectra and the value of $\Delta m_{32}^2$. \cite{zhan_determination_2008}
It was shown that for value of \todo{Stopped here for now. See V.L. JUNO chapter before getting back here}


The JUNO (Jiangmen Underground Neutrino Observatory) is a neutrino detection experiment located in China. Its main objective is the determination of the mass ordering at the 3-4$\sigma$ level in 6 years of data taking \cite{an_neutrino_2016}.

\section{Neutrinos physics in JUNO}

As said before, the main goals of JUNO are the determination of the NMO and the precise measurements of the oscillation parameters $\Delta m_{21}^2$, $\sin^2 2\theta_{12}$, $\Delta m_{32}^2$ and, with less precision, $\sin^2\theta_{13}$. \todo{Might remove this line if enough precision in precedent chapter/section}

\subsection{Reactor neutrino oscillation for NMO and precise measurements}

Previous works \cite{zhan_determination_2008,  zhan_experimental_2009} shows that oscillation parameters and the NMO can be observed by looking at the $\bnue$ disappearance spectrum coming from medium baseline nuclear reactor. This disappearance probability can be expressed as \cite{an_neutrino_2016} :
\begin{equation*}
  P(\bnue \rightarrow \bnue) = 1 - \sin^2 2\theta_{12} c^4_{13} sin^2 \frac{\Delta m^2_{21}L}{4E} - \sin^2 2\theta_{13} \bigg[ c_{12}^2 \sin^2 \frac{\Delta m_{31}^2 L}{4E} + s^2_{12} \sin^2 \frac{\Delta m_{32}^2 L}{4E} \bigg]
\end{equation*}
Where $s_{ij} = \sin \theta_{ij}$, $c_{ij} = \cos \theta_{ij}$, $E$ is the $\bnue$ energy and $L$ is the baseline.
We can see the sensitivity to the NMO in the dependency to $\Delta m_{32}^2$ and $\Delta m^2_{31}$ causing a phase shift of the spectrum (Figure \ref{fig:juno-spectrum-oscillation}).
By carefully fitting this spectrum, one can extract the NMO and the oscillation parameters.
\begin{figure}
  \centering
  \includegraphics[height=8cm]{images/juno/Spectrum-OscillationsOnly_dm2_31.png}
  \caption{Expected number of neutrinos event per MeV in JUNO after 6 years of data taking. The black curve shows the flux if there was no oscillation. The light gray curve shows the oscillation if only the solar terms are taken in account ($\theta_{12}$, $\Delta m_{21}^2$). The blue and red curve shows the spectrum in the case of, respectively, NO and IO. The dependency of the oscillation to the different parameters are schematized by the double sided arrows. We can see the NMO sensitivity by looking at the fine phase shift between the red and the blue curve.}
  \label{fig:juno-spectrum-oscillation}
\end{figure}

\subsubsection{Identification of the mass ordering}

To identify the mass ordering, we fit the neutrino energy spectrum under the two hypothesis of NO and IO. Those two fit give us two $\chi^2$, respectively $\chi^2_{NO}$ and $\chi^2_{IO}$. By computing the difference $\Delta \chi ^2 = \chi^2_{NO} - \chi^2_{IO}$ we can determine the most probable mass ordering: NO if $\Delta \chi^2 > 0$ and IO if $\Delta \chi^2 < 0$. Current studies shows that the expected sensitivity the mass ordering would be of $3.4 \sigma$ after 6 years of data taking in nominal setup\cite{an_neutrino_2016}.

\subsubsection{Precise measurement of the oscillations parameters}

The oscillations parameters $\theta_{12}$, $\theta_{13}$, $\Delta m^2_{21}$, $\Delta m^2_{31}$ are free parameters in the fit of the oscillation spectrum. The precision on those parameters have been estimated and are shown in figure \ref{fig:juno-param-precision}. Wee see that for $\theta_{12}$, $\Delta m^2_{21}$, $\Delta m^2_{31}$, precision at 6 years is better than the reference precision by an order of magnitude \cite{juno_collaboration_sub-percent_2022}

\begin{figure}[hb]
  \centering
  \includegraphics[width=\linewidth]{images/juno/oscillation_params_precision.png}
  \caption{A summary of precision levels fir the oscillation parameters. The reference value (PDG 2020 \cite{pdg2020}) is compared with 100 days, 6 years and 20 years of JUNO data taking.}
  \label{fig:juno-param-precision}
\end{figure}

\subsection{Other physics}

While reactor neutrinos are the main signal, JUNO will of course be sensitive to every neutrinos energetic enough to allow an IBD, and even some other more exotic channels.

\subsubsection{Geoneutrinos}

Geoneutrinos designate the antineutrinos coming from the decay of long-lived radioactive elements inside the Earth. The 1.8 MeV threshold necessary for the IBD makes it possible to measure geoneutrinos from $^238$U and $^232$Th decay chains. The studies of geoneutrinos can help refine the Earth crust models but is also necessary to characterise their signal, as they are a background to the mass ordering and oscillations parameters studies.

\subsubsection{Atmospheric neutrinos}

Atmospheric neutrinos are neutrinos originating from the decay of $\pi$ and $K$ particles that are produced in extensive air showers initiated by the interactions of cosmic rays with the Earth atmosphere. Earth is mostly transparent to neutrinos below the PeV energy, thus JUNO will be able to see neutrinos coming from all directions. Their baseline range is large (15km $\sim$ 13000km), they can have energy between 0.1 GeV and 10 TeV and will contain all neutrino and antineutrinos flavour.

\subsubsection{Beyond standard model neutrinos interactions}

\subsubsection{Supernovae burst neutrinos}

\subsubsection{Diffuse supernovae neutrinos background}

\subsubsection{Background in the neutrinos reactor spectrum}

\section{The JUNO detector}

\subsection{Central Detector (CD)}

\subsubsection{Acrylic containment sphere}

\subsubsection{Liquid scintillator}

\subsubsection{Large photo-multipliers (LPMTs)}

\subsubsection{Small photo-multipliers (SPMTs)}

\subsubsection{Data Acquisition System (DAQ)}

\subsubsection{Simulation}

\subsubsection{Software}

\todo{Expliquer comment le software fonctionne}


\subsection{Veto detector}

\subsubsection{Cherenkov in water pool}

\subsubsection{Top tracker}

\section{Calibration strategy}

\subsection{Energy scale calibration}

\subsection{Calibration system}

\subsection{Calibration program}

\section{Event selection and background rejection}

\todo{Explication de comment reconnaitre un IDB (OEC)}

\subsection{Fiducial volume}

\subsection{Muon tagging}

\section{State of the art of the IBD reconstruction}

\subsection{Interaction vertex reconstruction}

\subsection{Energy reconstruction}

\subsection{Particle identification}

\subsection{Machine learning for reconstruction}

\subsubsection{Vertex reconstruction}

\subsubsection{Energy reconstruction}

\section{JUNO sensitivity to NMO and precise measurements}

\subsection{Fitting procedure}
